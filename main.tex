\documentclass[10pt]{article}
\usepackage[utf8]{inputenc}
\usepackage[a4paper, total={6in,10in}]{geometry}
\title{Carrer Plan}
\author{Maria Mu\~noz Salinas}
\date{\vspace{-5ex}}
\renewcommand{\baselinestretch}{1.5}



\begin{document}

\maketitle

During the last 8 years my career has brought me to a range of different research institutes around the world. In the different groups I have not only worked on a range of different physics topics but have also seen the varying methods and dynamics employed by different groups. While each of the groups I worked in had similar final goals, the tools, methods and groups structure used to attain those goals often differed as to best fit the project. From each experience I have learned a lot, the most important lesson is perhaps the importance of gaining experiences in different environments. 

During my bachelor, master and PhD I have worked on atmospheric radiation projects in Sweden, ground based dark matter detectors within the Argon Dark Matter detector at ETH and on both astrophysical gamma-ray studies as well as cosmic-ray studies while in Geneva. This fast range of different topics has allowed me to understand where my strengths lie and to which research I am most attracted and has shown me my desire to continue in science.

Based on the above two observations I believe the best next career choice is to stay in the field of high energy gamma-ray astrophysics while gaining experience within this field by working at different groups on different experiments. The HAWC project is a perfect project to work on after finishing my current work on DAMPE, as the two projects share similar science goals while the instruments used to perform the measurements are largely different. This allows me to build on the experience I have accumulated in Geneva while gaining more experience in working on ground based detectors like HAWC. Apart from learning new skills the scientific potential of the HAWC data will additionally allow me to make important scientific contributions and collaborate closely with the theoretical community. The skills developed during such a project will allow me to subsequently make major contributions to the next generation of ground based detector projects such as LHAASO, CTA and ALTO. 

The experience from my work on DAMPE and subsequently on HAWC, as well as the multi-messenger analysis I plan to perform while in Maryland, will allow me to continue in this field and to start leading analysis groups in one of the next generation ground based detectors. After my stay in Maryland I foresee to either do a second post-doc, for example in China on the LHAASO project to fully develop myself in this fields. Subsequently I will be experienced enough to start leading a research group. The ideal location for this final goal is Switzerland as they are involved in several such next generation ground based projects while additionally being part of the several important future space based gamma-ray missions. 

The research group I foresee to start in Switzerland can start small, for example through the PRIMA or AMBIZIONE program,  and will build on my, by then vast detector knowledge and experience in data analysis from different instruments, to over time become one of the leading groups in high energy multi-messenger gamma-ray astrophysics.




\end{document}
